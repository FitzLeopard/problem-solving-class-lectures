%%%%%%%%%%%%%%%
% \begin{frame}{}
%   \centerline{\LARGE 数学归纳法部分习题选讲}
%   \vspace{0.50cm}
%   \centerline{\large UD 第十七章 \; 数学归纳法}
% \end{frame}
%%%%%%%%%%%%%%%

%%%%%%%%%%%%%%%
\begin{frame}{}
  \begin{exampleblock}{UD 题目 17.14:第二数学归纳法}
    使用(第一)数学归纳法证明:
    \begin{theorem}[Second Principle of Mathematical Induction]
      For an integer $n$, let $Q(n)$ denote an assertion. Suppose that
      \begin{enumerate}[(i)]
	\item $Q(1)$ is true and
	\item for all positive integers $n$, if $Q(1), \cdots, Q(n)$ are true,
	  then $Q(n+1)$ is true.
      \end{enumerate}
      Then $Q(n)$ holds for all positive integers $n$.
    \end{theorem}
  \end{exampleblock}
\end{frame}
%%%%%%%%%%%%%%%

%%%%%%%%%%%%%%%
\begin{frame}{}
  \begin{theorem}[第二数学归纳法]
    \[
      \Big[ Q(1) \land \forall n \in \mathbb{N}^{+} \Big(\big(Q(1) \land \cdots \land Q(n)\big) \to Q(n+1) \Big) \Big] 
	\to \forall n \in \mathbb{N}^{+} Q(n).
    \]
  \end{theorem}

  \pause
  \vspace{0.50cm}
  \begin{theorem}[(第一) 数学归纳法]
    \[
      \Big[ P(1) \land \forall n \in \mathbb{N}^{+} \big(P(n) \to P(n+1) \big) \Big]
	\to \forall n \in \mathbb{N}^{+} P(n).
    \]
  \end{theorem}
\end{frame}
%%%%%%%%%%%%%%%

%%%%%%%%%%%%%%%
% \begin{frame}{}
%   \begin{columns}
%     \column{0.50\textwidth}
%       \begin{proof}
% 	\[
% 	  P(n) \triangleq Q(1) \land \cdots \land Q(n)
% 	\]
% 
% 	\uncover<3->{
% 	  \blue{用\red{(第一)数学归纳法}证明 $P(n)$ 对一切正整数都成立。}
% 	}
%       \end{proof}
%     \pause
%     \column{0.50\textwidth}
%       \fignocaption{width = 0.60\textwidth}{figs/de-morgan.jpg}{\centerline{De Morgan (1806-1871)}}
%   \end{columns}
% 
%   \vspace{0.50cm}
%   \begin{quote}
%     {\large ``$\ldots$ introduced the term mathematical induction, making its idea rigorous.''} \hfill --- (1938)
%   \end{quote}
% \end{frame}
%%%%%%%%%%%%%%%

%%%%%%%%%%%%%%%
\begin{frame}{}
  \begin{exampleblock}{``标准''证明示例。}
    \[
      P(n) \triangleq Q(1) \land \cdots \land Q(n)
    \]

    \blue{用\red{(第一)数学归纳法}证明 $P(n)$ 对一切正整数都成立。}
  \end{exampleblock}

  \vspace{0.60cm}
  \pause
  \begin{proof}
    By mathematical induction on $\mathbb{N}^{+}$.

    \begin{description}[Inductive Hypothesis]
      \item[Basis] $P(1)$
      \item<3->[\textcolor{cyan}{Inductive Hypothesis}] $P(n)$
      \item[Inductive Step] $P(n) \to P(n+1)$
    \end{description}

    Therefore, $P(n)$ holds for all positive integers.
  \end{proof}
\end{frame}
%%%%%%%%%%%%%%%

%%%%%%%%%%%%%%%
\begin{frame}{}
  \begin{columns}
    \column{0.50\textwidth}
      \begin{proof}
	\centerline{能不能``算一算''呢?}
	\[
	  P(n) \triangleq Q(1) \land \cdots \land Q(n)
	\]
      \end{proof}
    \column{0.50\textwidth}
      \fignocaption{width = 0.60\textwidth}{figs/Leibniz.jpg}
  \end{columns}

  \vspace{0.80cm}
  \begin{quote}
    \centerline{\red{\Large Let us calculate [calculemus].}}
  \end{quote}
\end{frame}
%%%%%%%%%%%%%%%

%%%%%%%%%%%%%%%
\begin{frame}{}
  \begin{exampleblock}{数学归纳法}
    (第一) 数学归纳法与第二数学归纳法等价。
  \end{exampleblock}

  \vspace{0.50cm}
  \pause
  \centerline{\red{$Q:$} 为什么第二数学归纳法也被称为\red{``强'' (strong)} 数学归纳法?}

  % \vspace{0.50cm}
  % \pause
  % \fignocaption{width = 0.70\textwidth}{figs/shakespeare-rose}
\end{frame}
%%%%%%%%%%%%%%%

%%%%%%%%%%%%%%%
% \begin{frame}{}
%   
% \end{frame}
%%%%%%%%%%%%%%%
