%%%%%%%%%%%%%%%
\begin{frame}{}
  \begin{columns}
    \column<2->{0.30\textwidth}
      \fignocaption{width = 0.75\textwidth}{figs/fudan}
    \column<1->{0.35\textwidth}
      \fignocaption{width = 0.80\textwidth}{figs/logic-tag-cloud}
    \column<2->{0.30\textwidth}
      \fignocaption{width = 0.75\textwidth}{figs/engines-of-logic}
  \end{columns}
\end{frame}
%%%%%%%%%%%%%%%

%%%%%%%%%%%%%%%
\begin{frame}{}
  \centerline{\LARGE 习题选讲}

  \begin{align*}
    \text{\cyan{UD (第五章)}} \quad & \text{ 反证法 (Contradiction)} \\
    \text{\cyan{UD (第十七章)}} \quad & \text{ 数学归纳法 (Mathematical Induction)} \\
    \text{\cyan{ES (第二十四章)}} \quad & \text{ 鸽笼原理 (Pigeonhole Principle)}
  \end{align*}

  \pause
  \fignocaption{width = 0.40\textwidth}{figs/so-easy-ad}
\end{frame}
%%%%%%%%%%%%%%%

%%%%%%%%%%%%%%%
\begin{frame}{}
  \begin{exampleblock}{UD 题目 17.14:第二数学归纳法}
    使用(第一)数学归纳法证明第二数学归纳法。
  \end{exampleblock}

  \vspace{0.20cm}
  \begin{theorem}[Cantor Theorem]
    Let $A$ be a set. 

    If $f: A \to 2^{A}$, then $f$ is not onto.
  \end{theorem}

  \pause
  \fignocaption{width = 0.20\textwidth}{figs/sand-clock}
\end{frame}
%%%%%%%%%%%%%%%
