\section{Counterfeit Coin Problem}

%%%%%%%%%%%%%%%
\begin{frame}{The Original Counterfeit Coin Problem}
  \begin{quote}
    You have \blue{eight} similar coins and a beam balance.

    \blue{At most one} coin is counterfeit and hence underweight.

    How can you detect whether there is an underweight coin, 

    and if so, which one, using the balance only \blue{twice}? \\[8pt]

    \hfill --- E.D. Schell, 1945 (American Mathematical Monthly)
  \end{quote}
\end{frame}
%%%%%%%%%%%%%%%

%%%%%%%%%%%%%%%
\begin{frame}{The Counterfeit Coin Problem in Homework}
  \begin{quote}

    \hfill --- Problem 1.8 of UD
  \end{quote}
\end{frame}
%%%%%%%%%%%%%%%

%%%%%%%%%%%%%%%
\begin{frame}{Understanding the Problem}
  The \red{minimum} number of weighings $\dots$

  In the worst-case scenario

  Decision tree

  ``min-max''
\end{frame}
%%%%%%%%%%%%%%%

%%%%%%%%%%%%%%%
\begin{frame}{What Can We Do?}
  \centerline{Put equal numbers of coins on opposite sides of the balance.}

  Same?
\end{frame}
%%%%%%%%%%%%%%%

%%%%%%%%%%%%%%%
\begin{frame}{What is the First Step?}
  \[
    L = x \qquad R = x
  \]

  Possible outcomes:
  \begin{description}
    \item[Balanced]
    \item[$L$ Rises]
    \item[$R$ Rises]
  \end{description}
\end{frame}
%%%%%%%%%%%%%%%

%%%%%%%%%%%%%%%
\begin{frame}{Balanced: The ``Standard Coin'' Variation}
  Key point: G

\end{frame}
%%%%%%%%%%%%%%%

%%%%%%%%%%%%%%%
\begin{frame}{$L$ Rises: The ``Labelled Coin'' Variation}
  Key point: PH \& PL \& G

\end{frame}
%%%%%%%%%%%%%%%

%%%%%%%%%%%%%%%
\begin{frame}{A Special Case of the ``Labelled Coin'' Variation}
  \centerline{The counterfeit coin is known to be light.}

  Recursive algorithm: 
  \[
    \frac{1}{3}
  \]

  Lower bound: 
  \begin{quote}
    a single weighing of any sort cannot do better than trisection
  \end{quote}
\end{frame}
%%%%%%%%%%%%%%%

%%%%%%%%%%%%%%%
\begin{frame}{The ``Labelled Coin'' Variation}
  Recursive algorithm:
  \begin{quote}
    Whenever we place coins on the scale, 
    we must be sure to put \red{equal} number of PL (therefore PH) coins on the two sides.
  \end{quote}

  Lower bound:
  \begin{quote}
    cannot do better than in the ``Light Coin'' variation
  \end{quote}
\end{frame}
%%%%%%%%%%%%%%%

%%%%%%%%%%%%%%%
\begin{frame}{The ``Labelled Coin'' Variation in the 12 Coins Example}
\end{frame}
%%%%%%%%%%%%%%%

%%%%%%%%%%%%%%%
\begin{frame}{The ``Standard Coin'' Variation}
  \[
    M(n) = (3^n - 1) / 2
  \]
\end{frame}
%%%%%%%%%%%%%%%

%%%%%%%%%%%%%%%
\begin{frame}{}
\end{frame}
%%%%%%%%%%%%%%%
