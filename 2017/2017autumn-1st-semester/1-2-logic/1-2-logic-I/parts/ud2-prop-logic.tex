%%%%%%%%%%%%%%%
\begin{frame}{}
  \centerline{\LARGE 命题逻辑部分习题选讲}
  \vspace{0.50cm}
  \centerline{\large UD 第二章 \; 命题逻辑基础知识}
\end{frame}
%%%%%%%%%%%%%%%

%%%%%%%%%%%%%%%
\begin{frame}{}
  \begin{exampleblock}{题目 2.1: 前提、结论}
    \begin{center}
      if \\[8pt]
      whenever \\[12pt]
      \red{\large only if} (只有 $\cdots$, 才 $\cdots$; 除非 $\cdots$)
    \end{center}
  \end{exampleblock}

  \[
    p \text{ only if } q
  \]

  \pause
  \vspace{0.80cm}
  \begin{quote}
    \centerline{只有男足夺冠了/游戏打通关了,我才能安心学习。}
  \end{quote}

  \pause
  \begin{quote}
    \centerline{要想人不知,除非己莫为。}
  \end{quote}
\end{frame}
%%%%%%%%%%%%%%%

%%%%%%%%%%%%%%%
\begin{frame}{}
  \begin{exampleblock}{题目 2.5:命题逻辑中的语义}
    \[
      \big(P \to (\lnot R \lor Q)\big) \land R
    \]

    \vspace{0.40cm}
    \centerline{\red{\large 真值表 (truth table)}~\footnote{``T/F'' 是元语言中的概念,不是命题逻辑中的概念。}}
  \end{exampleblock}

  \pause

  \[
    p \to q
  \]

  \begin{quote}
    \centerline{如果男足夺冠了/游戏打通关了,我就安心学习。}
  \end{quote}

  \pause
  \begin{exampleblock}{题目 2.8:运算优先级}
    \[
      P \land Q \lor R
    \]
    \pause
    \centerline{(程序设计:短路求值)}
  \end{exampleblock}
\end{frame}
%%%%%%%%%%%%%%%

%%%%%%%%%%%%%%%
\begin{frame}{}
  \begin{exampleblock}{题目 2.6:否定}
    If the stars are green or white horse is shining,
    then the world is eleven feet wide.
  \end{exampleblock}

  \vspace{0.50cm}

  以下否定形式是否正确?\\[8pt]

  The stars are green, the white horse is shining, but the world is not eleven feet wide.
\end{frame}
%%%%%%%%%%%%%%%

%%%%%%%%%%%%%%%
\begin{frame}{}
  \begin{exampleblock}{题目 2.7:永真式 (Tautology)}
    \begin{enumerate}[(a)]
      \item $\lnot(\lnot P)$
      \item $\lnot(P \lor Q)$
      \item $\lnot(P \land Q)$
      \item $P \to Q$
    \end{enumerate}
  \end{exampleblock}

  \vspace{0.30cm}
  \begin{columns}
    \column{0.40\textwidth}
      对于 (a), 这个答案正确吗?
      \begin{enumerate}[(a)]
	\item $P$
      \end{enumerate}
    \pause
    \column{0.40\textwidth}
      \begin{itemize}
	\item DeMorgan's Law \\
	  (在程序设计中的应用)
	\item 蕴涵 (implication)
	  \[
	    (P \to Q) \leftrightarrow (\lnot P \lor Q)
	  \]
      \end{itemize}
  \end{columns}

  % \pause
  % \vspace{0.30cm}
  % \begin{exampleblock}{题目 2.10:等价}
  %   ``It snows or it is not sunny.''
  % \end{exampleblock}
\end{frame}
%%%%%%%%%%%%%%%

%%%%%%%%%%%%%%%
\begin{frame}{}
  \begin{exampleblock}{题目 2.11:诚实的人 vs. 说谎者}
    On a certain island,
    \begin{itemize}
      \item Each inhabitant is either a truth-teller or a liar (not both).
      \item A truth-teller always tells the truth and a liar always lies.
      \item Arnie and Barnie live on the island.
    \end{itemize}

    \begin{enumerate}[(a)]
      \item Arnie: ``If I am a truth-teller, then each person living on this island is either a truth-teller or a liar.''
      \item Arnie: ``If I am a truth-teller, then so is Barnie.''
    \end{enumerate}
  \end{exampleblock}

  \vspace{0.20cm}
  \pause
  \begin{enumerate}[(a)]
    \item Is Arnie a truth-teller or a liar?
    \item Can you tell what Arnie and Barnie are?
  \end{enumerate}

  \vspace{0.20cm}
  \pause
  \centerline{更重要的是, 你能\red{\LARGE ``算''}出来吗?}
  \pause
  \[
    (b) \; A \leftrightarrow (A \to B)
  \]
\end{frame}
%%%%%%%%%%%%%%%
