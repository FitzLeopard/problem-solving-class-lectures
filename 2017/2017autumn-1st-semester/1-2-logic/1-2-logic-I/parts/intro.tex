%%%%%%%%%%%%%%%
\begin{frame}{\uncover<2->{Gottfried Wilhelm Leibniz (莱布尼茨 1646 -- 1716)}}
  \fignocaption{width = 0.35\textwidth}{figs/Leibniz.jpg}
\end{frame}
%%%%%%%%%%%%%%%

%%%%%%%%%%%%%%%
\begin{frame}{``我有一个梦想 $\ldots$''}
  \begin{quote}
    建立一个能够涵盖人类思维活动的\red{\large ``通用符号演算系统''},\\
    让人们的思维方式变得像数学运算那样清晰。\\[8pt]

    一旦有争论, 不管是科学上的还是哲学上的,
    人们只要坐下来\red{\LARGE 算一算},
    就可以毫不费力地辨明谁是对的。
  \end{quote}

  \vspace{0.80cm}
  \begin{quote}
    \centerline{\red{\LARGE Let us calculate [calculemus].}}
  \end{quote}
\end{frame}
%%%%%%%%%%%%%%%

%%%%%%%%%%%%%%%
\begin{frame}{数理逻辑}
  \centerline{数理逻辑是一门使用数学的方法研究\red{\large ``推理''}的学科。}
  \vspace{0.80cm}

  \begin{columns}
    \column{0.40\textwidth}
    四个部分 (狭义):
      \begin{itemize}
	\item 集合论
	\item 模型论
	\item 递归论
	\item 证明论
      \end{itemize}
    \column{0.40\textwidth}
    命题逻辑与一阶谓词逻辑:
      \begin{itemize}
	\item 公理系统
	\item \red{\LARGE 推理规则}
	\item 语法与语义
	\item 可靠性与完全性
      \end{itemize}
  \end{columns}
\end{frame}
%%%%%%%%%%%%%%%

%%%%%%%%%%%%%%%
\begin{frame}{学习数理逻辑的三种途径}
  \begin{columns}
    \column{0.25\textwidth}
      \fignocaption{width = 0.85\textwidth}{figs/fudan.jpg}
    \column{0.10\textwidth}
      \fignocaption{width = 0.60\textwidth}{figs/vs.png}
    \column{0.25\textwidth}
      \fignocaption{width = 0.85\textwidth}{figs/uc.jpg}
    \column{0.10\textwidth}
      \fignocaption{width = 0.60\textwidth}{figs/vs.png}
    \column{0.25\textwidth}
      \fignocaption{width = 0.90\textwidth}{figs/what-logic.jpg}
  \end{columns}
\end{frame}
%%%%%%%%%%%%%%%

%%%%%%%%%%%%%%%
\begin{frame}{殊途不同归}
  \fignocaption{width = 0.25\textwidth}{figs/book-view.jpg}
\end{frame}
%%%%%%%%%%%%%%%