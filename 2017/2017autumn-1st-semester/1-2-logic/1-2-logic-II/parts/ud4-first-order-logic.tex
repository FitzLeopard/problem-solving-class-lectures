%%%%%%%%%%%%%%%
\begin{frame}{}
  \centerline{\LARGE 一阶谓词逻辑部分习题选讲}
  \vspace{0.50cm}
  \centerline{\large UD 第四章 \; 量词}
\end{frame}
%%%%%%%%%%%%%%%

%%%%%%%%%%%%%%%
\begin{frame}{}
  \begin{columns}
    \column{0.60\textwidth}
      \begin{quote}
	\uncover<2->{
	  逻辑是一项需要经过学习才能掌握的技能, 但是这项技能对你来说也是\red{天赋}的。\\[10pt]
	}

	\uncover<3->{
	  如果你不得不死记一条逻辑定律而毫不感到有\red{心灵上的碰撞}%
	  或者毫不领悟\red{为何此定律理应成立},\\
	  那么你也无法正确有效地使用它。\\[8pt]
	}

	\hfill --- ``Analysis'', Terrence Tao
      \end{quote}
    \column{0.30\textwidth}
      \fignocaption{width = 1.00\textwidth}{figs/terrence-tao-color.jpg}
  \end{columns}
\end{frame}
%%%%%%%%%%%%%%%

%%%%%%%%%%%%%%%
\begin{frame}{一阶谓词语言的语义}
  \[
    L = \set{<}
  \]

  \[
    \psi: \forall x \exists y \; (y < x)
  \]

  \vspace{0.30cm}
  \pause
  \centerline{\red{$Q:$} $\psi$ 是真是假?}

  \vspace{0.50cm}
  \pause
  \[
    \mathcal{U} = \mathbb{N}
  \]

  \pause
  \[
    \mathcal{U} = \mathbb{Z}
  \]
\end{frame}
%%%%%%%%%%%%%%%

%%%%%%%%%%%%%%%
\begin{frame}{一阶谓词语言中的重言式}
  \pause
  \[
    \Big(\forall y \lnot P(y) \to \lnot P(x)\Big) \to \Big(P(x) \to \exists y P(y)\Big)
  \]

  \[
    \Big(\forall x (\alpha \to \beta)\Big) \to (\forall x \alpha \to \forall x \beta)
  \]
\end{frame}
%%%%%%%%%%%%%%%

%%%%%%%%%%%%%%%
% \begin{frame}{一阶谓词逻辑的公理系统与推理规则}
% \end{frame}
%%%%%%%%%%%%%%%

%%%%%%%%%%%%%%%
\begin{frame}{学生反馈 (I)}
  \begin{quote}
    Suppose a statement restricts the variable $x$ to a proper subset $A$
    of the universe as in the statement form, $\cdots$

    \hfill --- ``Tips on Quantification'' (UD P51)
  \end{quote}
  
  \vspace{0.60cm}
  \begin{columns}
    \column{0.50\textwidth}
      ``For all $x \in A$, $p(x)$ holds.''
      \[
	\forall x \; \Big(x \in A \to P(x)\Big)
      \]

      \uncover<3->{
	\[
	  \red{\forall x \; \Big(x \in A \land P(x)\Big)}
	\]
      }
    \column{0.50\textwidth}
      ``For some $x \in A$, $p(x)$ holds.''
      \[
	\exists x \; \Big(x \in A \land P(x)\Big)
      \]

      \uncover<3->{
	\[
	  \red{\exists x \; \Big(x \in A \to P(x)\Big)}
	\]
      }
  \end{columns}

  \vspace{0.30cm}
  \centerline{\uncover<2->{\red{$Q:$} 为什么 $\forall$ 就要用 $\to$, 而 $\exists$ 就要用 $\land$?}}
\end{frame}
%%%%%%%%%%%%%%%

%%%%%%%%%%%%%%%
\begin{frame}{学生反馈 (II)}
  \begin{columns}
    \column{0.50\textwidth}
      ``For all $x \in A$, $p(x)$ holds.''
      \[
	\forall x \; \Big(x \in A \to P(x)\Big)
      \]

      \uncover<2->{
	\[
	  \blue{\forall x \in A. \; P(x)}
	\]
      }
    \column{0.50\textwidth}
      ``For some $x \in A$, $p(x)$ holds.''
      \[
	\exists x \; \Big(x \in A \land P(x)\Big)
      \]

      \uncover<2->{
	\[
	  \blue{\exists x \in A. \; P(x)}
	\]
      }
  \end{columns}

  \vspace{0.30cm}
  \centerline{\red{$Q:$} 在高中阶段,我们还经常用 $\forall x \in A/\exists x \in A$。现在还能这样写吗?}

  \pause
  \vspace{0.60cm}
  \centerline{By definition (shorthand).}
\end{frame}
%%%%%%%%%%%%%%%

%%%%%%%%%%%%%%%
\begin{frame}{}
  \begin{exampleblock}{题目 4.1:量词 $\forall$、$\exists$}
    \begin{enumerate}[(a)]
      \setcounter{enumi}{3}
      \item There exists an $x$ such that for some $y$ the equality $x = 2y$ holds.
      \item There exists an $x$ and a $y$ such that $x = 2y$.
    \end{enumerate}
  \end{exampleblock}

  \vspace{0.60cm}
  \pause
  你犯了下面这些``富有想象力的''错误了吗?

  \[
    \exists x \to \exists y, x = 2y
  \]

  \[
    \exists (x,y), x = 2y
  \]

  \[
    \exists x, y, x = 2y
  \]

  \[
    \exists x, y, \to x = 2y
  \]
\end{frame}
%%%%%%%%%%%%%%%

%%%%%%%%%%%%%%%
\begin{frame}{}
  \begin{exampleblock}{题目 4.5:量词的否定}
    \begin{enumerate}[(a)]
      \setcounter{enumi}{7}
      \item If $x \neq 0$, then there exists $y$ such that $xy = 1$.
    \end{enumerate}
  \end{exampleblock}

  \vspace{0.50cm}
  对于 $(h)$, 以下公式表述正确吗?
  \[
    \exists x \neq 0, \exists y (xy = 1)
  \]
\end{frame}
%%%%%%%%%%%%%%%

%%%%%%%%%%%%%%%
\begin{frame}{}
  \begin{exampleblock}{题目 4.5:量词的否定}
    \begin{enumerate}[(a)]
      \setcounter{enumi}{9}
      \item For all $\epsilon > 0$, there exists $\delta > 0$ such that 
	if $x$ is a real number with $|x - 1| < \delta$, then $|x^2 - 1| < \epsilon$.
    \end{enumerate}
  \end{exampleblock}

  \vspace{0.50cm}
  \pause
  \[
    \forall \epsilon > 0, \exists \delta > 0, (x \in R \land |x - 1| < \delta) \to |x^2 - 1| < \epsilon.
  \]

  \vspace{0.30cm}
  \pause
  否定形式为什么不是这样的?
  \[
    \exists \epsilon \red{\le} 0, \forall \delta \red{\le} 0, (x \in R \land |x - 1| < \delta) \land |x^2 - 1| \ge \epsilon.
  \]

  \pause
  \vspace{-0.30cm}
  \[
    \blue{(\lnot \forall x\, \alpha) \leftrightarrow (\exists x\, \lnot \alpha)}
  \]
  
  \pause
  \vspace{-0.30cm}
  \[
    \red{\Big(\lnot \forall x \in A.\, P(x)\Big) \leftrightarrow \Big(\exists x \in A.\, \lnot P(x)\Big)}
  \]
\end{frame}
%%%%%%%%%%%%%%%

%%%%%%%%%%%%%%%
\begin{frame}{}
  \begin{exampleblock}{题目 4.5:量词的否定}
    \begin{enumerate}[(a)]
      \setcounter{enumi}{10}
      \item For all real numbers $M$, there exists a real number $N$ such that $|f(n)| > M$
	for all $n > N$.
    \end{enumerate}
  \end{exampleblock}

  \vspace{0.30cm}
  \[
    \forall M \in R, \exists N \in R, \forall n > N, |f(n)| > M.
  \]
  
  \pause
  \vspace{0.30cm}
  \[
    \exists M \in R, \forall N \in R, \exists n > N, |f(n)| \le M.
  \]
\end{frame}
%%%%%%%%%%%%%%%

%%%%%%%%%%%%%%%
\begin{frame}{}
  \begin{exampleblock}{题目 4.7:量词与蕴含的否定}
    \[
      \forall x \Big(\big(x \in \mathbb{Z} \land \lnot\big(\exists y (y \in \mathbb{Z} \land x = 7y)\big)\big)
	\to \big(\exists z (z \in \mathbb{Z} \land x = 2z)\big)\Big).
    \]
    \vspace{-0.60cm}
    \begin{enumerate}[(a)]
      % \setcounter{enumi}{1}
      %   \item Read it out.
      \setcounter{enumi}{0}
        \item Negate it.
      % \setcounter{enumi}{2}
      %   \item Which statement is \red{\large true}, the original one or the negation?
    \end{enumerate}
  \end{exampleblock}

  % \pause
  % \vspace{0.50cm}
  % 以下``读法''正确吗?
  % \begin{quote}
  %   For all $x$, if $x$ is an integer and for any integer $y$ we have $x \neq 7y$,
  %   then there exists an integer $z$ such that $x = 2z$.
  % \end{quote}

  \vspace{0.50cm}
  \pause
  \red{$Q:$} 以下否定形式正确吗?
  \[
    \exists x \Big( \big(x \in \mathbb{Z} \land \big(\forall y (y \notin \mathbb{Z} \lor x \neq 7y)\big) \big) 
    \land \big(\forall z (z \notin \mathbb{Z} \lor x \neq 2z) \big) \Big)
  \]

  \vspace{0.20cm}
  \pause
  \red{$Q:$} 你能将原公式写成 $\forall x \in \mathbb{Z} \cdots$ 形式吗?
\end{frame}
%%%%%%%%%%%%%%%

%%%%%%%%%%%%%%%
\begin{frame}{}
  \begin{exampleblock}{题目 4.13:一阶谓词逻辑的推理规则(及其公式的语义)}
    Decide whether (3) is true \red{\large if} (1) and (2) are both true.
  \end{exampleblock}

  \pause
  \vspace{0.30cm}
  \red{$Q$:} 该如何理解这道题? 依据什么 ``decide'' 真假?
  
  \pause
  \begin{description}
    \item[逻辑知识] 
      \[
	(1) \land (2) \to (3)
      \]
    \pause
    \item[数学知识] ``True'' 是语义概念 
      \begin{itemize}
	\item 与选定的``结构''中的知识有关
      \end{itemize}
  \end{description}

  \pause
  \fignocaption{width = 0.30\textwidth}{figs/mass.png}
\end{frame}
%%%%%%%%%%%%%%%

%%%%%%%%%%%%%%%
\begin{frame}{}
  \begin{exampleblock}{题目 4.13:一阶谓词逻辑的推理规则(及其公式的语义)}
    Decide whether (3) is true \red{\large if} (1) and (2) are both true.

    \begin{enumerate}[(a)]
      \item 
	\begin{enumerate}[(1)]
	  \item Everyone who loves Bill loves Sam.
	  \item I don't love Sam.
	  \item I don't love Bill.
	\end{enumerate}
    \end{enumerate}
  \end{exampleblock}

  \vspace{0.50cm}
  \pause
  \centerline{\red{$Q$:} 如何在一阶谓词逻辑框架中``算出来''?}
\end{frame}
%%%%%%%%%%%%%%%

%%%%%%%%%%%%%%%
\begin{frame}{}
  \begin{exampleblock}{题目 4.13:一阶谓词逻辑的推理规则(及其公式的语义)}
    Decide whether (3) is true \red{\large if} (1) and (2) are both true.

    \begin{enumerate}[(a)]
      \setcounter{enumi}{1}
      \item 
	\begin{enumerate}[(1)]
	  \item If Susie goes to the ball in the red dress, I will stay home.
	  \item Susie went to the ball in the green dress.
	  \item I did not stay home.
	\end{enumerate}
    \end{enumerate}
  \end{exampleblock}

  \centerline{\red{$Q$}: 这是真的吗?}

  \pause
  \vspace{0.30cm}
  到底是真是假?
  \vspace{0.30cm}
  \begin{columns}
    \column{0.45\textwidth}
      \begin{itemize}
	\item (3) is true:

	Whether I stay at home or not, (3) is always true.
      \end{itemize}
    \column{0.45\textwidth}
      \begin{itemize}
        \item (3) is false:

	No matter what I do, the implication is always true.
      \end{itemize}
  \end{columns}

  \pause
  \vspace{0.50cm}
  \centerline{实际上,仅根据 (1)、(2), 我们无法判断 (3) 的真假。}
\end{frame}
%%%%%%%%%%%%%%%

%%%%%%%%%%%%%%%
\begin{frame}{}
  \begin{exampleblock}{题目 4.13:一阶谓词逻辑的推理规则(及其公式的语义)}
    Decide whether (3) is true \red{\large if} (1) and (2) are both true.

    \begin{enumerate}[(a)]
      \setcounter{enumi}{2}
      \item 
	\begin{enumerate}[(1)]
	  \item If $l$ is a positive real number, then there exists a real number $m$ such that $m > l$.
	  \item Every real number $m$ is less than $t$.
	  \item The real number $t$ is not positive.
	\end{enumerate}
    \end{enumerate}
  \end{exampleblock}

  \vspace{0.30cm}
  \pause
  如何形式化表达 (1)、(2)、(3)?
  \begin{enumerate}[(1)]
    \pause
    \item $\forall l$ 还是仅是 $l$?
    \pause
    \item $t$ 究竟是不是实数?
    \pause
    \item $R(t) \land P(t)$ 还是 $R(t) \to P(t)$?
  \end{enumerate}

  \pause
  \vspace{0.30cm}
  \centerline{现在,让我们来``算''一下吧。}
\end{frame}
%%%%%%%%%%%%%%%

%%%%%%%%%%%%%%%
\begin{frame}{}
  \begin{exampleblock}{题目 4.13:一阶谓词逻辑的推理规则(及其公式的语义)}
    Decide whether (3) is true \red{\large if} (1) and (2) are both true.

    \begin{enumerate}[(a)]
      \setcounter{enumi}{3}
      \item 
	\begin{enumerate}[(1)]
	  \item Every little breeze seems to whisper Louise or my name is Igor.
	  \item My name is Stewart.
	  \item Every little breeze seems to whisper Louise.
	\end{enumerate}
    \end{enumerate}
  \end{exampleblock}
\end{frame}
%%%%%%%%%%%%%%%

%%%%%%%%%%%%%%%
\begin{frame}{}
  \begin{exampleblock}{题目 4.13:一阶谓词逻辑的推理规则(及其公式的语义)}
    Decide whether (3) is true \red{\large if} (1) and (2) are both true.

    \begin{enumerate}[(a)]
      \setcounter{enumi}{4}
      \item 
	\begin{enumerate}[(1)]
	  \item There is a house on every street such that if that house is blue,
	    the one next to it is black.
	  \item There is no blue house on my street.
	  \item There is no black house on my street.
	\end{enumerate}
    \end{enumerate}
  \end{exampleblock}

  \vspace{0.30cm}
  \centerline{(1) 在说什么? 翻译成汉语是什么意思?}

  \pause
  \fignocaption{width = 0.40\textwidth}{figs/what.jpg}

  \pause
  \[
    \forall s\in S\, \exists h \in H 
    \Big(\text{On}(h,s) \land \big(\text{Blue}(h) \to \text{Black}\big(\text{next-to}(h)\big) \big)\Big)
  \]
\end{frame}
%%%%%%%%%%%%%%%

%%%%%%%%%%%%%%%
\begin{frame}{}
  \begin{exampleblock}{题目 4.13:一阶谓词逻辑的推理规则(及其公式的语义)}
    Decide whether (3) is true \red{\large if} (1) and (2) are both true.

    \begin{enumerate}[(a)]
      \setcounter{enumi}{5}
      \item Let $x$ and $y$ be real numbers.
	\begin{enumerate}[(1)]
	  \item If $x > 5$, then $y < 1/5$.
	  \item We know $y = 1$.
	  \item So $x \le 5$.
	\end{enumerate}
    \end{enumerate}
  \end{exampleblock}

  \pause
  \vspace{0.80cm}
  \centerline{\red{$Q$}: 在推理过程中, 我们用到了哪些数学知识 (\blue{非逻辑}知识)?}
\end{frame}
%%%%%%%%%%%%%%%

%%%%%%%%%%%%%%%
\begin{frame}{}
  \begin{exampleblock}{题目 4.13:一阶谓词逻辑的推理规则(及其公式的语义)}
    Decide whether (3) is true \red{\large if} (1) and (2) are both true.

    \begin{enumerate}[(a)]
      \setcounter{enumi}{6}
      \item Let $M$ and $n$ be real numbers.
	\begin{enumerate}[(1)]
	  \item If $n > M$, then $n^2 > M^2$.
	  \item We know $n < M$.
	  \item So $n^2 \le M^2$.
	\end{enumerate}
    \end{enumerate}
  \end{exampleblock}

  \begin{columns}[t]
    \pause
    \column{0.30\textwidth}
      \begin{itemize}
	\item (3) is false: 
	  \[
	    n = -2, \; M = -1
	  \]
      \end{itemize}
    \pause
    \column{0.30\textwidth}
      \begin{itemize}
        \item (3) is true:
	\begin{align*}
	  (1)\; & n > 0 \\
	  (2)\; & 0 < n < M
	\end{align*}
      \end{itemize}
    \pause
    \column{0.30\textwidth}
      \begin{itemize}
        \item 无法判断
	\[
	  (1) \land (2) \to (3)
	\]
      \end{itemize}
  \end{columns}
  \pause
  \begin{columns}
    \column{0.45\textwidth}
      \fignocaption{width = 0.50\textwidth}{figs/yun.jpg}
    \column{0.45\textwidth}
      \fignocaption{width = 0.40\textwidth}{figs/Mn-qrcode.png}
  \end{columns}
  % \centerline{\url{https://math.stackexchange.com/q/2471687/51434}}
\end{frame}
%%%%%%%%%%%%%%%

%%%%%%%%%%%%%%%
\begin{frame}{}
  \begin{exampleblock}{题目 4.13:一阶谓词逻辑的推理规则(及其公式的语义)}
    Decide whether (3) is true \red{\large if} (1) and (2) are both true.

    \begin{enumerate}[(a)]
      \setcounter{enumi}{7}
      \item Let $x,y$, and $z$ be real numbers.
	\begin{enumerate}[(1)]
	  \item If $y > x$ and $y > 0$, then $y > z$.
	  \item We know that $y \le z$.
	  \item Then $y \le x$ or $y \le 0$.
	\end{enumerate}
    \end{enumerate}
  \end{exampleblock}
\end{frame}
%%%%%%%%%%%%%%%

