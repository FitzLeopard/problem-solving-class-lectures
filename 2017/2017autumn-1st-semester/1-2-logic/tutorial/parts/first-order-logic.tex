%%%%%%%%%%%%%%%
\begin{frame}{}
  \centerline{\LARGE 一阶谓词逻辑部分习题选讲}
\end{frame}
%%%%%%%%%%%%%%%

%%%%%%%%%%%%%%%
\begin{frame}{}
  \begin{exampleblock}{题目 4.1:量词 $\forall$、$\exists$}
    \begin{enumerate}[(a)]
      \setcounter{enumi}{3}
      \item There exists an $x$ such that for some $y$ the equality $x = 2y$ holds.
      \item There exists an $x$ and a $y$ such that $x = 2y$.
    \end{enumerate}
  \end{exampleblock}

  \vspace{0.60cm}
  \pause
  对于 $(d)$, 这个公式正确吗?
  \[
    \exists x \to \exists y, x = 2y
  \]
  \pause
  对于 $(e)$, 以下两个公式正确吗?
  \[
    \exists (x,y), x = 2y
  \]
  \vspace{-0.30cm}
  \[
    \exists x, y, x = 2y
  \]
\end{frame}
%%%%%%%%%%%%%%%

%%%%%%%%%%%%%%%
\begin{frame}{}
  \begin{exampleblock}{题目 4.5:量词的否定}
    \begin{enumerate}[(a)]
      \setcounter{enumi}{9}
      \item 
	\[
	  \forall \epsilon > 0, \exists \delta > 0, (x \in R \land |x - 1| < \delta) \to |x^2 - 1| < \epsilon.
	\]
      \item
	\[
	  \forall M \in R, \exists N \in R, \forall n > N, |f(n)| > M.
	\]
    \end{enumerate}
  \end{exampleblock}

  \vspace{0.50cm}
  \pause
  对于 $(j)$, 以下否定形式正确吗?
  \[
      \exists \epsilon > 0, \forall \delta > 0, (x \in R \land |x - 1| \ge \delta) \land |x^2 - 1| < \epsilon.
  \]

  \pause
  对于 $(k)$, 以下否定形式正确吗?
  \[
    \exists M \in R, \forall N \in R, \exists n > N, |f(n)| > M.
  \]
\end{frame}
%%%%%%%%%%%%%%%

%%%%%%%%%%%%%%%
\begin{frame}{}
  \begin{exampleblock}{题目 4.7:量词与蕴含的否定}
    \[
      \forall x, \Big(x \in \mathbb{Z} \land \lnot\big(\exists y, (y \in \mathbb{Z} \land x = 7y)\big)
	\land \big(\forall z, (z \in \mathbb{Z} \land x = 2z)\big)\Big).
    \]

    \uncover<3->{
      \begin{enumerate}[(a)]
	\setcounter{enumi}{2}
	\item Which statement is \red{\LARGE true}, the original one or the negation?
      \end{enumerate}
    }
  \end{exampleblock}

  \vspace{0.30cm}
  \uncover<2->{
    以下否定形式正确吗?
  }

  \uncover<4->{
    \centerline{一阶谓词逻辑公式的\red{\LARGE 语义}:如何定义``真假值''?}
  }
\end{frame}
%%%%%%%%%%%%%%%

%%%%%%%%%%%%%%%
\begin{frame}{}
\end{frame}
%%%%%%%%%%%%%%%

%%%%%%%%%%%%%%%
\begin{frame}{}
\end{frame}
%%%%%%%%%%%%%%%

%%%%%%%%%%%%%%%
\begin{frame}{}
\end{frame}
%%%%%%%%%%%%%%%

%%%%%%%%%%%%%%%
\begin{frame}{}
\end{frame}
%%%%%%%%%%%%%%%
