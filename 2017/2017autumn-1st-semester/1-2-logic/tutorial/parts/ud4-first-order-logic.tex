%%%%%%%%%%%%%%%
\begin{frame}{}
  \centerline{\LARGE 一阶谓词逻辑部分习题选讲}
  \vspace{0.50cm}
  \centerline{\large UD 第四章 \; 量词}
\end{frame}
%%%%%%%%%%%%%%%

%%%%%%%%%%%%%%%
\begin{frame}{}
  \begin{exampleblock}{题目 4.1:量词 $\forall$、$\exists$}
    \begin{enumerate}[(a)]
      \setcounter{enumi}{3}
      \item There exists an $x$ such that for some $y$ the equality $x = 2y$ holds.
      \item There exists an $x$ and a $y$ such that $x = 2y$.
    \end{enumerate}
  \end{exampleblock}

  \vspace{0.60cm}
  \pause
  对于 $(d)$, 这个公式正确吗?
  \[
    \exists x \to \exists y, x = 2y
  \]
  \pause
  对于 $(e)$, 以下两个公式正确吗?
  \[
    \exists (x,y), x = 2y
  \]
  \vspace{-0.30cm}
  \[
    \exists x, y, x = 2y
  \]
\end{frame}
%%%%%%%%%%%%%%%

%%%%%%%%%%%%%%%
\begin{frame}{}
  \begin{exampleblock}{题目 4.5:量词的否定}
    \begin{enumerate}[(a)]
      \setcounter{enumi}{9}
      \item 
	\[
	  \forall \epsilon > 0, \exists \delta > 0, (x \in R \land |x - 1| < \delta) \to |x^2 - 1| < \epsilon.
	\]
      \item
	\[
	  \forall M \in R, \exists N \in R, \forall n > N, |f(n)| > M.
	\]
    \end{enumerate}
  \end{exampleblock}

  \vspace{0.50cm}
  \pause
  对于 $(j)$, 以下否定形式正确吗?
  \[
      \exists \epsilon > 0, \forall \delta > 0, (x \in R \land |x - 1| \ge \delta) \land |x^2 - 1| < \epsilon.
  \]

  \pause
  对于 $(k)$, 以下否定形式正确吗?
  \[
    \exists M \in R, \forall N \in R, \exists n > N, |f(n)| > M.
  \]
\end{frame}
%%%%%%%%%%%%%%%

%%%%%%%%%%%%%%%
\begin{frame}{}
  \begin{exampleblock}{题目 4.7:量词与蕴含的否定}
    \[
      \forall x \Big(x \in \mathbb{Z} \land \lnot\big(\exists y (y \in \mathbb{Z} \land x = 7y)\big)
	\land \big(\forall z (z \in \mathbb{Z} \land x = 2z)\big)\Big).
    \]
    \vspace{-0.60cm}
    \begin{enumerate}[(a)]
      \setcounter{enumi}{1}
        \item Read it out.
      \setcounter{enumi}{0}
        \item Negate it.
      \setcounter{enumi}{2}
	\item Which statement is \red{\LARGE true}, the original one or the negation?
    \end{enumerate}
  \end{exampleblock}

  \pause
  \vspace{0.20cm}
  以下``读法''正确吗?
  \begin{quote}
    For all $x$, if $x$ is an integer and for any integer $y$ we have $x \neq 7y$,
    then there exists an integer $z$ such that $x = 2z$.
  \end{quote}

  \pause
  以下否定形式正确吗?
  \[
    \exists x \Big( \big(x \in \mathbb{Z} \land \big(\forall y (y \notin \mathbb{Z} \lor x \neq 7y)\big) \big) 
    \land \big(\forall z (z \notin \mathbb{Z} \lor x \neq 2z) \big) \Big)
  \]

  \pause
  一阶谓词逻辑公式的\red{\large 语义}:如何定义``真假值''?
\end{frame}
%%%%%%%%%%%%%%%

%%%%%%%%%%%%%%%
\begin{frame}{}
  \begin{exampleblock}{题目 4.13:一阶谓词逻辑的推理规则(及其公式的语义)}
    Decide whether (3) is true \red{\large if} (1) and (2) are both true.
  \end{exampleblock}

  \pause
  \vspace{0.50cm}
  你是如何理解这道题的?
  
  \begin{enumerate}
    \item ``True'' 是什么意思?
    \item 如何 Decide?
      \begin{itemize}
	\item 假设(3)为真, 结合(1)、(2),没有产生矛盾,就说明 (3) 是真的吗?
      \end{itemize}
  \end{enumerate}
\end{frame}
%%%%%%%%%%%%%%%

%%%%%%%%%%%%%%%
\begin{frame}{}
  \begin{exampleblock}{题目 4.13:一阶谓词逻辑的推理规则(及其公式的语义)}
    Decide whether (3) is true \red{\large if} (1) and (2) are both true.

    \begin{enumerate}[(a)]
      \item 
	\begin{enumerate}[(1)]
	  \item Everyone who loves Bill loves Sam.
	  \item I don't love Sam.
	  \item I don't love Bill.
	\end{enumerate}
    \end{enumerate}
  \end{exampleblock}

  \vspace{0.50cm}
  \pause
  \centerline{\red{问题}:如何在一阶谓词逻辑框架中推理出该结论?}
\end{frame}
%%%%%%%%%%%%%%%

%%%%%%%%%%%%%%%
\begin{frame}{}
  \begin{exampleblock}{题目 4.13:一阶谓词逻辑的推理规则(及其公式的语义)}
    Decide whether (3) is true \red{\large if} (1) and (2) are both true.

    \begin{enumerate}[(a)]
      \setcounter{enumi}{1}
      \item 
	\begin{enumerate}[(1)]
	  \item If Susie goes to the ball in the red dress, I will stay home.
	  \item Susie went to the ball in the green dress.
	  \item I did not stay home.
	\end{enumerate}
    \end{enumerate}
  \end{exampleblock}

  \centerline{是真的吗?}

  \pause
  \vspace{0.30cm}
  到底是真是假?
  \vspace{0.30cm}
  \begin{columns}
    \column{0.45\textwidth}
      \begin{itemize}
	\item (3) is true:

	Whether I stay at home or not, (3) is always true.
      \end{itemize}
    \column{0.45\textwidth}
      \begin{itemize}
        \item (3) is false:

	No matter what I do, the implication is always true.
      \end{itemize}
  \end{columns}

  \pause
  \vspace{0.50cm}
  实际上,仅根据 (1)、(2), 我们无法判断 (3) 的真假 (尽管 (3) 是个命题)。
\end{frame}
%%%%%%%%%%%%%%%

%%%%%%%%%%%%%%%
\begin{frame}{}
  \begin{exampleblock}{题目 4.13:一阶谓词逻辑的推理规则(及其公式的语义)}
    Decide whether (3) is true \red{\large if} (1) and (2) are both true.

    \begin{enumerate}[(a)]
      \setcounter{enumi}{2}
      \item 
	\begin{enumerate}[(1)]
	  \item If $l$ is a positive real number, then there exists a real number $m$ such that $m > l$.
	  \item Every real number $m$ is less than $t$.
	  \item The real number $t$ is not positive.
	\end{enumerate}
    \end{enumerate}
  \end{exampleblock}

  \vspace{0.30cm}
  \begin{enumerate}
    \item 是真是假?
      \pause
    \item 如何形式化表达?\\
      \pause
      (2) 中 $t$ 究竟是不是实数?
      \pause
    \item ``让我们算一算''
  \end{enumerate}
\end{frame}
%%%%%%%%%%%%%%%

%%%%%%%%%%%%%%%
\begin{frame}{}
  \begin{exampleblock}{题目 4.13:一阶谓词逻辑的推理规则(及其公式的语义)}
    Decide whether (3) is true \red{\large if} (1) and (2) are both true.

    \begin{enumerate}[(a)]
      \setcounter{enumi}{3}
      \item 
	\begin{enumerate}[(1)]
	  \item Every little breeze seems to whisper Louise or my name is Igor.
	  \item My name is Stewart.
	  \item Every little breeze seems to whisper Louise.
	\end{enumerate}
    \end{enumerate}
  \end{exampleblock}
\end{frame}
%%%%%%%%%%%%%%%

%%%%%%%%%%%%%%%
\begin{frame}{}
  \begin{exampleblock}{题目 4.13:一阶谓词逻辑的推理规则(及其公式的语义)}
    Decide whether (3) is true \red{\large if} (1) and (2) are both true.

    \begin{enumerate}[(a)]
      \setcounter{enumi}{4}
      \item 
	\begin{enumerate}[(1)]
	  \item There is a house on every street such that if that house is blue,
	    the one next to it is black.
	  \item There is no blue house on my street.
	  \item There is no black house on my street.
	\end{enumerate}
    \end{enumerate}
  \end{exampleblock}

  \vspace{0.50cm}
  \centerline{(1) 在说什么?}
\end{frame}
%%%%%%%%%%%%%%%

%%%%%%%%%%%%%%%
\begin{frame}{}
  \begin{exampleblock}{题目 4.13:一阶谓词逻辑的推理规则(及其公式的语义)}
    Decide whether (3) is true \red{\large if} (1) and (2) are both true.

    \begin{enumerate}[(a)]
      \setcounter{enumi}{5}
      \item Let $x$ and $y$ be real numbers.
	\begin{enumerate}[(1)]
	  \item If $x > 5$, then $y < 1/5$.
	  \item We know $y = 1$.
	  \item So $x \le 5$.
	\end{enumerate}
    \end{enumerate}
  \end{exampleblock}
\end{frame}
%%%%%%%%%%%%%%%

%%%%%%%%%%%%%%%
\begin{frame}{}
  \begin{exampleblock}{题目 4.13:一阶谓词逻辑的推理规则(及其公式的语义)}
    Decide whether (3) is true \red{\large if} (1) and (2) are both true.

    \begin{enumerate}[(a)]
      \setcounter{enumi}{6}
      \item Let $M$ and $n$ be real numbers.
	\begin{enumerate}[(1)]
	  \item If $n > M$, then $n^2 > M^2$.
	  \item We know $n < M$.
	  \item So $n^2 \le M^2$.
	\end{enumerate}
    \end{enumerate}
  \end{exampleblock}

  \vspace{0.30cm}
  到底哪个是正确的? 好像都有道理哦。
  \begin{columns}[t]
    \column{0.30\textwidth}
      \begin{itemize}
	\item (3) is false: 
	  \[
	    n = -2, \; M = -1
	  \]
      \end{itemize}
    \column{0.30\textwidth}
      \begin{itemize}
        \item (3) is true:
	\[
	  (1): n > 0;\; (2): 0 < n < M
	\]
      \end{itemize}
    \column{0.20\textwidth}
      \begin{itemize}
        \item 无法判断
      \end{itemize}
  \end{columns}

  \vspace{0.60cm}
  \pause
  \centerline{\url{https://math.stackexchange.com/q/2471687/51434}}
\end{frame}
%%%%%%%%%%%%%%%

%%%%%%%%%%%%%%%
\begin{frame}{}
  \begin{exampleblock}{题目 4.13:一阶谓词逻辑的推理规则(及其公式的语义)}
    Decide whether (3) is true \red{\large if} (1) and (2) are both true.

    \begin{enumerate}[(a)]
      \setcounter{enumi}{7}
      \item Let $x,y$, and $z$ be real numbers.
	\begin{enumerate}[(1)]
	  \item If $y > x$ and $y > 0$, then $y > z$.
	  \item We know that $y \le z$.
	  \item Then $y \le x$ or $y \le 0$.
	\end{enumerate}
    \end{enumerate}
  \end{exampleblock}
\end{frame}
%%%%%%%%%%%%%%%

%%%%%%%%%%%%%%%
\begin{frame}[noframenumbering]{}
  \fignocaption{width = 0.50\textwidth}{figs/thankyou.png}
\end{frame}
%%%%%%%%%%%%%%%
