\section{Euclid's Algorithm}

%%%%%%%%%%%%%%%%%%%%
\begin{frame}{Worst-case analysis of Euclid's algorithm}
  \begin{exampleblock}{(TC 31.2--5)}
	\begin{enumerate}
	  \item If $a > b \ge 0$, $\textsc{Euclid}(a,b)$ makes $\le 1 + \log_{\phi}b$ recursive calls.
	\end{enumerate}
  \end{exampleblock}

  \pause
  \[
	\text{Lam\'{e}'s theorem: } a > b \ge 1, b < F_{k+1} \implies r < k.
  \]

  \pause
  \[
	k = 2 + \log_{\phi}b
  \]

  \pause
  \[
	\text{To prove } b < F_{3 + \log_{\phi}b}.
  \]

  \pause
  \[
	F_k = \frac{\phi^{k} - \hat{\phi^{k}}}{\sqrt{5}} = [\frac{\phi^{k}}{\sqrt{5}}] \ge \frac{\phi^{k}}{\sqrt{5}}
  \]
\end{frame}
%%%%%%%%%%%%%%%%%%%%
\begin{frame}{Worst-case analysis of Euclid's algorithm}
  \begin{exampleblock}{(TC 31.2--5)}
	\begin{enumerate}
	  \setcounter{enumi}{1}
	  \item Improve this bound to $1 + \log_{\phi}(\frac{b}{(a,b)})$.
	\end{enumerate}
  \end{exampleblock}

  \pause
  \[
	(a,b) = (a,b) \cdot (\frac{a}{(a,b)}, \frac{b}{(a,b)})
  \]

  \pause
  \vspace{-0.20cm}
  \begin{columns}
	\column{0.20\textwidth}
	  \begin{align*}
		&(16,12) \\
		&=(12,4) \\
		&=(4,0) \\
		&=4
	  \end{align*}
	\column{0.20\textwidth}
	  \begin{align*}
		&(4,3) \\
		&=(3,1) \\
		&=(1,0) \\
		&=1
	  \end{align*}
  \end{columns}
  
  \pause
  \[
	\textsc{Euclid}(a,b) \leftrightarrow \textsc{Euclid}(\frac{a}{(a,b)}, \frac{b}{(a,b)})
  \]
\end{frame}
%%%%%%%%%%%%%%%%%%%%
\begin{frame}{Worst-case analysis of Euclid's algorithm}
  \begin{exampleblock}{(TC 31.2--5)}
	\begin{enumerate}
	  \setcounter{enumi}{1}
	  \item Improve this bound to $1 + \log_{\phi}(\frac{b}{(a,b)})$.
	\end{enumerate}
  \end{exampleblock}

  \[
	\textsc{Euclid}(a,b) \leftrightarrow \textsc{Euclid}(\frac{a}{(a,b)}, \frac{b}{(a,b)})
  \]

  \[
	\textsc{Euclid}(b, a \bmod b) \leftrightarrow \textsc{Euclid}(\frac{b}{(a,b)}, \frac{a}{(a,b)} \bmod \frac{b}{(a,b)})
  \]

  \[
	\frac{a}{(a,b)} \bmod \frac{b}{(a,b)} = \frac{a \bmod b}{(a,b)}
  \]
\end{frame}
%%%%%%%%%%%%%%%%%%%%
\begin{frame}{Worst-case analysis of Euclid's algorithm}
  \begin{exampleblock}{(TC 31.2--5)}
	\begin{enumerate}
	  \setcounter{enumi}{1}
	  \item Improve this bound to $1 + \log_{\phi}(\frac{b}{(a,b)})$.
	\end{enumerate}
  \end{exampleblock}

  \begin{Lemma}[Generalization of Lemma 31.10]
	If $a > b \ge 1, d = (a,b)$ and $\textsc{Euclid}(a,b)$ performs $k \ge 1$ recursive calls,
	then $a \ge dF_{k+2}$ and $b \ge dF_{k+1}$.
  \end{Lemma}
\end{frame}
%%%%%%%%%%%%%%%%%%%%
\begin{frame}{Average-case analysis of Euclid's algorithm}
  \[
	T(m,0) = 0; \quad T(m,n) = 1 + T(n, m \bmod n) \; n \ge 1
  \]
  
  \pause
  When $m$ is chosen at random:
  \[
	T_n = \frac{1}{n} \sum_{0 \le k < n} T(k,n)
  \]

  \pause
  Assume that, for $0 \le k < n$, $(n \bmod k)$ is ``random'':
  \[
	T_n \approx 1 + \frac{1}{n} (T_0 + T_1 + \cdots + T_{n-1}) \pause = 1 + \frac{1}{2} + \cdots + \frac{1}{n} = H_n \approx \ln n + O(1)
  \]

  \pause
  \begin{alertblock}{Reference}
	``The Art of Computer Programming, Vol 2: Seminumerical Algorithms (Section 4.5.3)'' by Donald E. Knuth, 3rd edition.
  \end{alertblock}
\end{frame}
%%%%%%%%%%%%%%%%%%%%
