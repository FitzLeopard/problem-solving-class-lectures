\section{Modular Arithmetic}

\begin{frame}{Cancellation in modular arithmetic}
  \begin{exampleblock}{(TC 31.4--2)}
	\[
	  ad \equiv bd \pmod{n} \centernot\implies a \equiv b \pmod{n} 
	\]

	\[
	  \uncover<3->{ad \equiv bd \pmod{n}, \textcolor{red}{d \bot n} \implies a \equiv b \pmod{n}}
	\]
  \end{exampleblock}

  \[
	\uncover<2->{3 \cdot 2 \equiv 5 \cdot 2 \pmod{4} \quad 3 \not\equiv 5 \pmod{4}}
  \]
\end{frame}
%%%%%%%%%%%%%%%%%%%%
\begin{frame}{Changing the modulus}
  \[
	3 \cdot 2 \equiv 5 \cdot 2 \pmod{4} \quad 3 \not\equiv 5 \pmod{4}
	\quad \textcolor{blue}{3 \equiv 5 \pmod{2}}
  \]

  \pause
  \[
	ad \equiv bd \pmod{nd} \iff a \equiv b \pmod{n} \quad (d \neq 0)
  \]

  \pause
  \[
	\boxed{(a \bmod n) d = ad \bmod nd \quad (\text{distributive law})}
  \]

  \pause
  \[
	ad \equiv bd \pmod{n} \iff a \equiv b \pmod{\frac{n}{(d,n)}}
  \]
\end{frame}
%%%%%%%%%%%%%%%%%%%%
\begin{frame}{Changing the modulus}
  \[
	n = n_1n_2\cdots n_k
  \]

  \pause
  \[
	a \equiv b \pmod{n} \implies a \equiv b \pmod{n_i}
  \]

  \pause
  \[
	a \equiv b \pmod{100} \implies a \equiv b \pmod{20} \implies a \equiv b \pmod{5}
  \]
\end{frame}
%%%%%%%%%%%%%%%%%%%%
\begin{frame}{Changing the modulus}
  \[
	n = n_1n_2\cdots n_k
  \]

  \pause
  \[
	a \equiv b \pmod{n_1}, a \equiv b \pmod{n_2} \iff a \equiv b \pmod{\lcm(n_1, n_2)}
  \]

  \pause
  \[
	a \equiv b \pmod{n_1}, a \equiv b \pmod{n_2} \iff a \equiv b \pmod{n_1n_2}, \text{ if } n_1 \bot n_2
  \]

  \pause
  \[
	\forall 1 \le i \le k, a \equiv b \pmod{n_i} \iff a \equiv b \pmod{n}, \text{ if } n_i \bot n_j
  \]
\end{frame}
%%%%%%%%%%%%%%%%%%%%
