\section{Rotational Symmetries of Cube}

%%%%%%%%%%%%%%%%%%%%
\begin{frame}{$C \cong S_4$}
  \begin{itemize}
    \item Order of $1$: id ($\# = 1$)
    \item Order of $4$: face-to-face
      \begin{align*}
	 f_{td} = (1\;2\;3\;4) \quad f_{td}^{2} = (1\;3) (2\;4)  \quad f_{td}^{3} = (1\;4\;3\;2) \\
	 f_{lr} = (1\;2\;4\;3) \quad f_{lr}^{2} = (1\;4) (2\;3)  \quad f_{lr}^{3} = (1\;3\;4\;2) \\
	 f_{fb} = (1\;4\;2\;3) \quad f_{fb}^{2} = (1\;2) (3\;4)  \quad f_{fb}^{3} = (1\;3\;2\;4)
      \end{align*}
  \end{itemize}
\end{frame}
%%%%%%%%%%%%%%%%%%%%
\begin{frame}{$C \cong S_4$}
  \begin{itemize}
    \item Order of $3$: vertex-to-vertex
      \begin{align*}
        v_{1} = (2\;3\;4) \quad v_{1}^{2} = (2\;4\;3) \\
        v_{2} = (1\;4\;3) \quad v_{2}^{2} = (1\;3\;4) \\
        v_{3} = (1\;2\;4) \quad v_{3}^{2} = (1\;4\;2) \\
        v_{4} = (1\;2\;3) \quad v_{4}^{2} = (1\;3\;2) \\
      \end{align*}
    \item Order of $2$: edge-to-edge
      \begin{align*}
        e_{12} = (1\;2) \quad e_{13} = (1\;3) \quad e_{14} = (1\;4) \\
        e_{23} = (2\;3) \quad e_{24} = (2\;4) \quad e_{34} = (3\;4)
      \end{align*}
  \end{itemize}
\end{frame}
%%%%%%%%%%%%%%%%%%%%
\begin{frame}{Subgroups of $S_4$}
  Possible orders: $1 \quad 2 \quad 3 \quad 4 \quad 6 \quad 8 \quad 12 \quad 24$

  \begin{itemize}
    \item $|H| = 1$: $\# = 1$
    \item $|H| = 24$: $\# = 1$
    \item $|H| = 2$: $\# = 6 + 3 = 9$
    \item $|H| = 3$: $\# = 4$
  \end{itemize}
\end{frame}
%%%%%%%%%%%%%%%%%%%%
\begin{frame}{Subgroups of order 4}
  \begin{itemize}
    \item $H \cong \mathbb{Z}_{4}$: $\# = 3$
    \item $H \cong K_{4} = \set{e, a, b, c} (a^2 = b^2 = c^2)$
    \begin{align*}
      \set{(1), (1\; 2), (3\; 4), (1\; 2) (3\; 4)}\\
      \set{(1), (1\; 3), (2\; 4), (1\; 3) (2\; 4)}\\
      \set{(1), (1\; 4), (2\; 3), (1\; 4) (2\; 3)}\\
      \textcolor{red}{\set{(1), (1\; 2) (1\; 3), (2\; 4), (1\; 4) (2\; 3)}}
    \end{align*}
  \end{itemize}

  \[
    \# = 3 + 4 = 7
  \]
\end{frame}
%%%%%%%%%%%%%%%%%%%%
\begin{frame}{Subgroups of order 6}
  \[
    H \ncong \mathbb{Z}_{6}
  \]

  \[
    H \cong S_3 = \set{1, r, r^2, s, rs, r^2s} \quad (r^3 = 1, s^2 = 1)
  \]

  \centerline{Figure here.}

  \begin{theorem}
    There are only 4 subgroups of order 6 in $S_4$.
  \end{theorem}

  \[
    r = (1\;3\;2), \quad s = (1\;3)
  \]

  \centerline{What does $srs = r^{-1}$ mean?}
\end{frame}
%%%%%%%%%%%%%%%%%%%%
\begin{frame}{Subgroups of order 8}
  \[
    H \ncong \mathbb{Z}_{8}
  \]

  \[
    H \ncong \mathbb{Z}_{2} \times \mathbb{Z}_{2} \times \mathbb{Z}_{2}
  \]

  \[
    H \ncong \mathbb{Z}_{4} \times \mathbb{Z}_{2}
  \]

  \[
    H \ncong Q_8: \implies |H| \ge 9
  \]

  \[
    H \cong D_4 = \set{1, r, r^2, r^3, s, rs, r^2s, r^3s} \quad (r^4 = 1, s^2 = 1)
  \]

  \centerline{Figure here.}

  \begin{theorem}
    There are only 3 subgroups of order 8 of $S_4$.
  \end{theorem}

  \[
    r = (1\;2\;4\;3), \quad s = (2\;3)
  \]

  \centerline{What does $srs = r^{-1}$ mean?}
\end{frame}
%%%%%%%%%%%%%%%%%%%%
\begin{frame}{Subgroups of order 12}
  \[
    H \cong \mathbb{Z}_{12}, \mathbb{Z}_{6} \times \mathbb{Z}_{2}, D_{6}, A_{4}, Dic_{12}
  \]

  \[
    H \cong A_4
  \]

  \centerline{Figure here.}

  \begin{theorem}
    There is only one subgroup of order $12$ in $S_4$.
  \end{theorem}

  \begin{proof}
  \end{proof}
\end{frame}
%%%%%%%%%%%%%%%%%%%%
