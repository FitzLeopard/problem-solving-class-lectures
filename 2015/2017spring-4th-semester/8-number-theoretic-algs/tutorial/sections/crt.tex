\section{Chinese Remainder Theorem}

%%%%%%%%%%%%%%%%%%%%
\begin{frame}{Chinese Remainder Theorem (CRT)}
  \begin{theorem}[CRT]
	\[
	  n_1, \ldots, n_k; \quad a_1, \ldots, a_k
	\]

	\[ 
	  n_i \bot n_j \quad i \neq j, \quad n = n_1n_2\cdots n_k 
	\]

    \[
	  \exists! a\; (\textcolor{red}{0 \le a < n}): a \equiv a_i \pmod{n_i}.
	\]
  \end{theorem}

  \begin{proof}[Proof for uniqueness]
	\[
	  a \equiv a' \pmod{n_i} \implies n \mid a - a'.
	\]
  \end{proof}
\end{frame}
%%%%%%%%%%%%%%%%%%%%
\begin{frame}{History of CRT}
  \begin{quote}
  \end{quote}
\end{frame}
%%%%%%%%%%%%%%%%%%%%
\begin{frame}{Proof of CRT (1)}
  \begin{proof}[Nonconstructive proof]
	\begin{align*}
	  &f: [0,n) \to \prod_{1 \le i \le k} [0,a_i) \\
	  &f: a \mapsto \big( a \pmod{n_1}, \dots, a \pmod{n_k} \big)
	\end{align*}

	\begin{itemize}
	  \item $f$ is one-to-one.
	  \item $f$ is onto.
		\[
		  \exists a: f(a) = (a_1, \dots, a_k).
		\]
	\end{itemize}
  \end{proof}
\end{frame}
%%%%%%%%%%%%%%%%%%%%
\begin{frame}{Proof of CRT (2)}
  \begin{align*}
	a &\equiv a_1 \pmod{n_1} \\
	a &\equiv a_2 \pmod{n_2}
  \end{align*}
\end{frame}
%%%%%%%%%%%%%%%%%%%%
\begin{frame}{Proof of CRT (3)}
  \begin{proof}[Constructive proof]
	\begin{enumerate}
	  \item $x \equiv 1 \pmod{n_i}, \quad x \equiv 0 \pmod{n_j}\; (i \neq j)$
		\[
		  x = M_i(M_i^{-1} \pmod{n_i}) \implies x \equiv M_iM_i^{-1} \pmod{m}
		\]
	  \item $x \equiv a_i \pmod{n_i}, \quad x \equiv 0 \pmod{n_j}\; (i \neq j)$
		\[
		  x \equiv a_i M_iM_i^{-1} \pmod{m}
		\]
	  \item $a \equiv a_i \pmod{n_i}$
		\[
		  a \equiv \sum_{1 \le i \le k} a_i M_iM_i^{-1} \pmod{m}
		\]
	\end{enumerate}
  \end{proof}
\end{frame}
%%%%%%%%%%%%%%%%%%%%
\begin{frame}{CRT}
  Meaning of Figure 31.3

  $\equiv 1$ and $\equiv 0$ elsewhere
\end{frame}
%%%%%%%%%%%%%%%%%%%%
\begin{frame}{The $\varphi$ function}
  \begin{theorem}[The $\varphi$ function]
	\[
	  m \bot n \implies \varphi(mn) = \varphi(m) \varphi(n)
	\]
  \end{theorem}

  \begin{proof}
	\begin{gather*}
	  U_{m} = \set{a \bmod m, (a,m) = 1}, U_{n} = \set{a \bmod n, (a,n) = 1}, \\
	  U_{mn} = \set{c \bmod mn, (c, mn) = 1}
	\end{gather*}

	\begin{align*}
	  f: U_{mn} &\to U_m \times U_n \\
	  f(c \bmod mn) &= (c \bmod m, c \bmod n).
	\end{align*}
  \end{proof}
\end{frame}
%%%%%%%%%%%%%%%%%%%%
\begin{frame}{The $\varphi$ function}
  \begin{theorem}[The $\varphi$ function]
	\[
	  \varphi(p^k) = p^k - p^{k-1}
	\]

	\[
	  \varphi(n) = n \prod_{p \mid n} (1 - frac{1}{p})
	\]
  \end{theorem}
\end{frame}
%%%%%%%%%%%%%%%%%%%%
\begin{frame}{Secret sharing using the CRT}
  \begin{definition}[$(k,n)$ threshold secret sharing scheme]
	$(2,3)$ secret sharing:
	\fignocaption{width = 0.35\textwidth}{figs/secret-sharing-geometry.png}
  \end{definition}
\end{frame}
%%%%%%%%%%%%%%%%%%%%
\begin{frame}{Secret sharing using the CRT}
  \begin{enumerate}
	\item Choose $m_i$:
	  \[
		m_1 < m_2 < \cdots < m_n, \quad m_i \bot m_j, \quad \prod_{i=n-k+2}^{n} m_i < \prod_{i=1}^{k} m_i
	  \]
	\item Choose the secret $S$:
	  \[
		\prod_{i=n-k+2}^{n} m_i < S < \prod_{i=1}^{k} m_i
	  \]
	\item Compute the shares:
	  \[
		s_i = S \bmod m_i
	  \]
  \end{enumerate}
\end{frame}
%%%%%%%%%%%%%%%%%%%%
\begin{frame}{CRT with non-pairwise coprime moduli}
\end{frame}
%%%%%%%%%%%%%%%%%%%%
\begin{frame}{Application?}
\end{frame}
%%%%%%%%%%%%%%%%%%%%
