\section{Chinese Remainder Theorem}

%%%%%%%%%%%%%%%%%%%%
\begin{frame}{Chinese Remainder Theorem (CRT)}
  \begin{theorem}[CRT]
	\[
	  n_1, \ldots, n_k; \quad a_1, \ldots, a_k
	\]

	\[ 
	  n_i \bot n_j \quad i \neq j, \quad n = n_1n_2\cdots n_k 
	\]

    \[
	  \exists! a\; (\textcolor{red}{0 \le a < n}): a \equiv a_i \pmod{n_i}.
	\]
  \end{theorem}

  \begin{proof}[Proof for uniqueness]
	\[
	  a \equiv a' \pmod{n_i} \implies n \mid a - a'.
	\]
  \end{proof}
\end{frame}
%%%%%%%%%%%%%%%%%%%%
\begin{frame}{History of CRT}
  \begin{quote}
  \end{quote}
\end{frame}
%%%%%%%%%%%%%%%%%%%%
\begin{frame}{Proof of CRT (1)}
  \begin{proof}[Nonconstructive proof]
	\begin{align*}
	  &f: [0,n) \to \prod_{1 \le i \le k} [0,a_i) \\
	  &f: a \mapsto \big( a \bmod n_1, \dots, a \bmod n_k \big)
	\end{align*}

	\begin{itemize}
	  \item $f$ is one-to-one.
	  \item $f$ is onto.
		\[
		  \exists a: f(a) = (a_1, \dots, a_k).
		\]
	\end{itemize}
  \end{proof}
\end{frame}
%%%%%%%%%%%%%%%%%%%%
\begin{frame}{Proof of CRT (2)}
  \begin{proof}[Constructive proof by induction]
	\begin{align}
	  a &\equiv a_1 \pmod{n_1} \\
	  a &\equiv a_2 \pmod{n_2}
	\end{align}

	\pause
	\[
	  (1) \implies a = a_1 + n_1 y
	\]
  \end{proof}

  % \pause
  % \begin{align*}
  %   a &= a_1 + n_1y \\
  %   n_1y &\equiv a_2 - a_1 \pmod{n_2} \\
  %   y &\equiv M_2^{-1}(a_2 - a_1) \pmod{n_2} \\
  %   n_1y &\equiv M_2M_2^{-1}(a_2 - a_1) \textcolor{red}{\pmod{n_1n_2}} \\
  %   x &\equiv a_1 + M_2M_2^{-1}(a_2 - a_1) \pmod{n_1n_2} \\
  %     &\equiv a_1M_1M_1^{-1} + a_2M_2M_2^{-1} \pmod{n_1n_2}
  % \end{align*}
\end{frame}
%%%%%%%%%%%%%%%%%%%%
\begin{frame}{Proof of CRT (3)}
  \begin{proof}[Constructive proof by induction]
	\begin{align}
	  a &\equiv a_1 \pmod{n_1} \\
	  a &\equiv a_2 \pmod{n_2}
	\end{align}

	\[
	  n_1 \bot n_2 \implies n_1n_1' + n_2n_2' = 1
	\]

	\pause
	\[
	  x = a_1n_1n_1' + a_2n_2n_2' \pmod{n_1n_2}
	\]
  \end{proof}
\end{frame}
%%%%%%%%%%%%%%%%%%%%
\begin{frame}{Proof of CRT (4)}
  \begin{proof}[Constructive proof]
	\begin{enumerate}[<+->]
	  \item $x \equiv 1 \pmod{n_i}, \quad x \equiv 0 \pmod{n_j}\; (i \neq j)$
		\[
		  x = M_i(M_i^{-1} \bmod n_i) \implies x = M_iM_i^{-1} \pmod{n}
		\]
	  \item $x \equiv a_i \pmod{n_i}, \quad x \equiv 0 \pmod{n_j}\; (i \neq j)$
		\[
		  x = a_i M_iM_i^{-1} \pmod{n}
		\]
	  \item $a \equiv a_i \pmod{n_i}, \forall 1 \le i \le k$
		\[
		  a = \sum_{1 \le i \le k} a_i M_iM_i^{-1} \pmod{n}
		\]
	\end{enumerate}
  \end{proof}
\end{frame}
%%%%%%%%%%%%%%%%%%%%
\begin{frame}{Proof of CRT (5)}
  \begin{proof}[More efficient constructive proof]
	\begin{alertblock}{Reference}
	  ``The Residue Number System'' by Garner, 1959.
	\end{alertblock}

	\begin{alertblock}{Reference}
	  ``The Art of Computer Programming, Vol 2: Seminumerical Algorithms (Section 4.3.2)'' by Donald E. Knuth, 3rd edition.
	\end{alertblock}
  \end{proof}
\end{frame}
%%%%%%%%%%%%%%%%%%%%
\begin{frame}{Operations over CRT}
  \[
	a \leftrightarrow (a_1, a_2, \ldots, a_n)
  \]

  \begin{align*}
	a \pm b &\leftrightarrow (a_1 \pm b_1, a_2 \pm b_2, \ldots, a_n \pm b_n) \\
	a \times b &\leftrightarrow (a_1 \times b_1, a_2 \times b_2, \ldots, a_n \times b_n) \\
  \end{align*}

  \pause
  \vspace{-0.30cm}

  \begin{exampleblock}{TC 31.5--3}
	\[
	 a \leftrightarrow (a_1, a_2, \ldots, a_n), (a,n) = 1 \implies a^{-1} \leftrightarrow (a_1^{-1}, a_2^{-2}, \ldots, a_n^{-1})
	\]
  \end{exampleblock}

  \pause
  \begin{proof}
	\begin{equation*}
	  a^{-1} \equiv a_i^{-1} \pmod{n_i} \pause \impliedby \begin{cases}
		a \equiv a_i \pmod{n_i} \\
		(a,n) = 1
	  \end{cases}
	\end{equation*}
  \end{proof}
\end{frame}
%%%%%%%%%%%%%%%%%%%%
\begin{frame}{The $\varphi$ function}
  \begin{theorem}[The $\varphi$ function]
	\begin{align*}
	  \varphi(p) &= p - 1 \\
	  \varphi(p^k) &= p^k - p^{k-1}
	\end{align*}
  \end{theorem}

  \pause

  \[
	n = \prod_{i=1}^{r} {p_i}^{k_i}
  \]

  \pause
  \[
	\textcolor{red}{m \bot n \implies \varphi(mn) = \varphi(m) \varphi(n)}
  \]

  \pause
  \[
	\varphi(n) = \prod_{i=1}^{r} \varphi(p_i^{k_i}) 
	    \pause = \prod_{i=1}^{r} (p_i^{k_i} - p_{i}^{k_i - 1})
		\pause = \prod_{i=1}^{r} p_i^{k_i} (1 - \frac{1}{p_i})
		\pause = n \prod_{i=1}^{r} (1 - \frac{1}{p_i})
  \]
\end{frame}
%%%%%%%%%%%%%%%%%%%%
\begin{frame}{The $\varphi$ function}
  \begin{theorem}[The $\varphi$ function]
	\[
	  m \bot n \implies \varphi(mn) = \varphi(m) \varphi(n)
	\]
  \end{theorem}

  \begin{proof}
	\begin{gather*}
	  U_{m} = \set{a \bmod m, (a,m) = 1}, U_{n} = \set{a \bmod n, (a,n) = 1}, \\
	  U_{mn} = \set{c \bmod mn, (c, mn) = 1}
	\end{gather*}

	\begin{align*}
	  f: U_{mn} &\to U_m \times U_n \\
	  f(c \bmod mn) &= (c \bmod m, c \bmod n).
	\end{align*}
  \end{proof}
\end{frame}
%%%%%%%%%%%%%%%%%%%%
\begin{frame}{Secret sharing using the CRT}
  \begin{definition}[$(k,n)$-threshold secret sharing scheme]
	$(2,3)$-secret sharing:
	\fignocaption{width = 0.30\textwidth}{figs/secret-sharing-geometry.png}
  \end{definition}

  \pause
  \begin{alertblock}{Reference}
	``How to Share a Secret'' by Mignotte, 1982.
  \end{alertblock}
\end{frame}
%%%%%%%%%%%%%%%%%%%%
\begin{frame}{Secret sharing using the CRT}
  \begin{enumerate}[<+->]
	\item Choose $m_i$:
	  \[
		m_1 < m_2 < \cdots < m_n, \quad m_i \bot m_j, \quad \prod_{i=n-k+2}^{n} m_i < \prod_{i=1}^{k} m_i
	  \]
	\item Choose the secret $S$:
	  \[
		\prod_{i=n-k+2}^{n} m_i < S < \prod_{i=1}^{k} m_i
	  \]
	\item Compute the shares:
	  \[
		s_i = S \bmod m_i
	  \]
  \end{enumerate}
\end{frame}
%%%%%%%%%%%%%%%%%%%%
\begin{frame}{Solving the system of congruences}
  \begin{exampleblock}{(TC 31.5--2)}
	\[
	  \begin{cases}
		x \equiv 1 \pmod{9} \\
		x \equiv 2 \pmod{8} \\
		x \equiv 3 \pmod{7}
	  \end{cases}
	\]
  \end{exampleblock}
\end{frame}
%%%%%%%%%%%%%%%%%%%%
\begin{frame}{Solving the system of congruences}
  \begin{exampleblock}{}
	\[
	  19x \equiv 556 \pmod{1155}
	\]
  \end{exampleblock}
\end{frame}
%%%%%%%%%%%%%%%%%%%%
\begin{frame}{Solving the system of congruences}
  \begin{exampleblock}{CRT with non-pairwise coprime moduli}
	\[
	  \begin{cases}
		x \equiv 3 \pmod{8} \\
		x \equiv 11 \pmod{20} \\
		x \equiv 1 \pmod{15} \\
	  \end{cases}
	\]
  \end{exampleblock}
\end{frame}
%%%%%%%%%%%%%%%%%%%%
