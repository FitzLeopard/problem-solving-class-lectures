\section{Primes}

%%%%%%%%%%%%%%%%%%%%
\begin{frame}{Pairwise relatively prime}
  \begin{exampleblock}{(TC 31.2--9)}
	\begin{gather*}
	  n_1, n_2, n_3, n_4 \text{ are pairwise relatively prime} \\
	  \iff \\
	  \gcd(n_1n_2, n_3n_4) = \gcd(n_1n_3, n_2n_4) = 1
	\end{gather*}
  \end{exampleblock}
\end{frame}
%%%%%%%%%%%%%%%%%%%%
\begin{frame}{Pairwise relatively prime}
  \begin{exampleblock}{(TC 31.2--9)}
	\begin{gather*}
	  n_1, n_2, \dots, n_k \text{ are pairwise relatively prime} \\
	  \iff \\
	  \text{a set of } \lceil \lg k \rceil \text{ pairs of numbers derived from the } n_i \text{ are relatively prime}.
	\end{gather*}
  \end{exampleblock}

  \pause
  \[
	\binom{k}{2} = \Theta(k^2)	\quad (\text{complete graph})
  \]

  \pause
  \[
	\textcolor{red}{\gcd(\fbox{$1_L$}, \fbox{$1_R$}) 
	= gcd(\fbox{$2_L$}, \fbox{$2_R$}) 
	= \cdots 
	= gcd(\fbox{$\lceil \lg k \rceil_L$}, \fbox{$\lceil \lg k \rceil_R$}) = 1}
  \]

  \pause
  \begin{gather*}
    % k = 4: \quad gcd(n_1n_2, n_3n_4) = gcd(n_1n_3, n_2n_4) = 1 \\
	k = 3: \quad gcd(n_1, n_2n_3) = gcd(n_2, n_3) = 1 \\
	k = 2: \quad gcd(n_1, n_2) = 1
  \end{gather*}
\end{frame}
%%%%%%%%%%%%%%%%%%%%
\begin{frame}{Pairwise relatively prime: divide-and-conquer}
  % \begin{center}
  %     \begin{tikzpicture}
	\Tree [.\node (1234567) {$n_1n_2n_3n_4n_5n_6n_7$};
	  [.\node (123) {$n_1n_2n_3$};
		[.\node (1) {$n_1$}; ] 
		  [.\node (23) {$n_2n_3$};
			[.\node (2) {$n_2$}; ]
			[.\node (3) {$n_3$}; ]]] 
	  [.\node (4567) {$n_4n_5n_6n_7$};
		[.\node (45) {$n_4n_5$};
		  [.\node (4) {$n_4$}; ] 
		  [.\node (5) {$n_5$}; ]]
		[.\node (67) {$n_6n_7$};
		  [.\node (6) {$n_6$}; ] 
		  [.\node (7) {$n_7$}; ]]]]

	\begin{pgfonlayer}{background}
	  % \node () [ellipse, draw, fit = (1234567), inner sep = 0pt] {};
	  \pause
	  \node () [rectangle, draw, fit = (2) (3), inner sep = 0pt] {};
	  \node () [rectangle, draw, fit = (4) (5), inner sep = 0pt] {};
	  \node () [rectangle, draw, fit = (6) (7), inner sep = 0pt] {};
	  \node () [rectangle, draw, fit = (1) (23), inner sep = 0pt] {};
	  \node () [rectangle, draw, fit = (45) (67), inner sep = 0pt] {};
	  \node () [rectangle, draw, fit = (123) (4567), inner sep = 0pt] {};
	\end{pgfonlayer}
  \end{tikzpicture}

  % \end{center}

  \fignocaption{width = 0.40\textwidth}{figs/coprime7.pdf}

  \pause
  \begin{equation*}
	\begin{cases}
	  T(1) = 0 \\
	  T(k) = 2T(\frac{k}{2}) + 1
	\end{cases} \pause \implies T(k) = k - 1
  \end{equation*}

  \pause
  \[
	T_k = k-1: (n_i, n_{i+1}n_{i+2}\cdots n_{k}) \quad \forall 1 \le i < k
  \]
\end{frame}
%%%%%%%%%%%%%%%%%%%%
\begin{frame}{Pairwise relatively prime: smarter combination}
  \centerline{TODO: figure here.}

  \begin{equation*}
	\begin{cases}
	  T(1) = 0 \\
	  T(k) = T(\frac{k}{2}) + 1
	\end{cases}\pause \implies T(k) = \lceil \lg k \rceil
  \end{equation*}
\end{frame}
%%%%%%%%%%%%%%%%%%%%
\begin{frame}{Pairwise relatively prime: the dividing pattern}
  \[
	n_{\textcolor{red}{0}}, n_1, n_2, \ldots, n_{k-1}
  \]
\end{frame}
%%%%%%%%%%%%%%%%%%%%
\begin{frame}{Can we do even better?}
  \[
	T(k) \ge \lceil \lg k \rceil.
  \]

  \pause
  \centerline{Prove by (strong) mathematical induction.}

  \pause
  \begin{align*}
	T(k) &\ge 1 + T(\lceil \frac{k}{2} \rceil) \\
		&\ge 1 + \lceil \lg \lceil \frac{k}{2} \rceil \rceil \\
		&= \lceil \lg k \rceil
  \end{align*}
\end{frame}
%%%%%%%%%%%%%%%%%%%%
\begin{frame}{Biclique covering}
  \centerline{Covering a complete graph with few complete bipartite subgraphs.}

  \fignocaption{width = 0.38\textwidth}{figs/biclique-covering-google.png}
\end{frame}
%%%%%%%%%%%%%%%%%%%%
\begin{frame}{Biclique covering: rethinking the first divide-and-conquer}
  \[
	T(k) = k-1
  \]

  \pause
  \centerline{\emph{edge-disjoint} biclique partition}

  \pause
  \vspace{0.20cm}
  \begin{alertblock}{Reference for $T(k) \ge k-1$}
	``On the Addressing Problem for Loop Switching'' by Graham and Pollak, 1971. 
  \end{alertblock}

  \pause
  \vspace{0.30cm}
  \begin{alertblock}{Reference for \emph{weighted} biclique partition}
	``Covering a Graph by Complete Bipartite Graphs'' by P. Erdos and L. Pyber, 1997.
  \end{alertblock}
\end{frame}
%%%%%%%%%%%%%%%%%%%%
