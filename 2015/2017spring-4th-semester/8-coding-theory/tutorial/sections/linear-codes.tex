\section{Linear Codes}

%%%%%%%%%%%%%%%%%%%%
\begin{frame}{Definition of linear code}
  \begin{definition}[Linear code]
	A linear code $C$ of length $n$ is a linear subspace of the vector space $\mathbb{F}_2^n$.
  \end{definition}

  \[
	c_1 \in C, c_2 \in C \implies c_1 + c_2 \in C
  \]

  \[
	\begin{aligned}
	  d(C) &= \min \set{w(c_1 + c_2) \mid c_1 \neq c_2, c_1, c_2 \in C} \\
	  &{\blue{=}} \min \set{w(c) \mid c \neq 0, c \in C}
	\end{aligned}
  \]
\end{frame}
%%%%%%%%%%%%%%%%%%%%
\begin{frame}{Definition of linear code}
  \begin{exampleblock}{Problem TJ--8.18}
	Let $C$ be a linear code.

	Show that either the $i$-th coordinates in the codewords of $C$ are all zeros
	or exactly half of them are zeros.
  \end{exampleblock}
\end{frame}
%%%%%%%%%%%%%%%%%%%%
\begin{frame}{Definition of linear code}
  \begin{exampleblock}{Problem TJ--8.19}
	Let $C$ be a linear code.

	Show that either every codeword has even weight
	or exactly half of them have even weight.
  \end{exampleblock}

  \[
	\text{Parity: } w(c_1) + w(c_2) \emph{ vs. } w(c_1 + c_2)
  \]
\end{frame}
%%%%%%%%%%%%%%%%%%%%
\begin{frame}{Definition of linear code}
  \begin{definition}[Linear code]
	An $(n,k)$ linear code $C$ of length $n$ and rank $k$ is a linear subspace with dimension $k$ of the vector space $\mathbb{F}_2^n$.
  \end{definition}

  \[
	\text{Basis: } c_1, c_2, \dots, c_k
  \]

  \[
	c_i = \alpha_1 c_1 + \alpha_2 c_2 + \cdots + \alpha_k c_k
  \]

  \[
	|C| = 2^k
  \]
\end{frame}
%%%%%%%%%%%%%%%%%%%%
\begin{frame}{Generator matrix}
  \begin{definition}[Generator matrix]
	A matrix $G_{n \times k}$ is a generator matrix for an $(n,k)$ linear code $C$ if 
	
	\[
	  C = \text{Col}(G)
	\]
  \end{definition}

  \[
	\text{rank}(G) = k
  \]

  \[
	G_{(n \times k)} \cdot d_{k \times 1} = c_{n \times 1} \in C
  \]

  \[
	G(c_1 + c_2) = G(c_1) + G(c_2)
  \]
\end{frame}
%%%%%%%%%%%%%%%%%%%%
\begin{frame}{Generator matrix for Hamming code $(7,4)$}
  \begin{columns}
	\column{0.45\textwidth}
	  \fignocaption{width = 0.60\textwidth}{figs/hamming74-venn.png}
	\column{0.45\textwidth}
	  \[
		G = \begin{bmatrix}
		  1 & 0 & 0 & 0 \\
		  0 & 1 & 0 & 0 \\
		  0 & 0 & 1 & 0 \\
		  0 & 0 & 0 & 1 \\
		  1 & 1 & 0 & 1 \\
		  0 & 1 & 1 & 1 \\
		  1 & 0 & 1 & 1 \\
		\end{bmatrix}
	  \]

  \end{columns}

  \[
	G \cdot \begin{pmatrix}
	  1 \\ 0 \\ 1 \\ 0
	\end{pmatrix}
	= 
  \]
\end{frame}
%%%%%%%%%%%%%%%%%%%%
\begin{frame}{Standard generator matrix}
  \begin{exampleblock}{Problem TJ--8.7}
	Generator matrices are NOT unique.
  \end{exampleblock}

  \begin{definition}[Standard generator matrix]
	A generator matrix $G_{n \times k}$ is standard if

	\[
	  G_{n \times k} = \begin{bmatrix}
		I_k \\ A_{(n-k) \times k}
	  \end{bmatrix}
	\]
  \end{definition}
\end{frame}
%%%%%%%%%%%%%%%%%%%%
\begin{frame}{From generator matrix to parity-check matrix}
  \[
	G \cdot \begin{pmatrix}
      d_1 \\ d_2 \\ d_3 \\ d_4
	\end{pmatrix}
	= \begin{bmatrix}
	  1 & 0 & 0 & 0 \\
	  0 & 1 & 0 & 0 \\
	  0 & 0 & 1 & 0 \\
	  0 & 0 & 0 & 1 \\
	  1 & 1 & 0 & 1 \\
	  0 & 1 & 1 & 1 \\
	  1 & 0 & 1 & 1 \\
	\end{bmatrix} 
	\cdot \begin{pmatrix}
      d_1 \\ d_2 \\ d_3 \\ d_4
	\end{pmatrix}
	= \begin{pmatrix}
      d_1 \\ d_2 \\ d_3 \\ d_4 \\
	  p_1 = d_1 + d_2  \qquad + d_4 \\
	  p_2 = \qquad d_2 + d_3 + d_4 \\
	  p_3 = d_1 \qquad + d_3 + d_4 \\
	\end{pmatrix}
  \]
\end{frame}
%%%%%%%%%%%%%%%%%%%%
\begin{frame}{From generator matrix to parity-check matrix}
  \begin{gather*}
	d_1 + d_2  \qquad + d_4 + p_1 \qquad\qquad = 0 \\
	\qquad d_2 + d_3 + d_4 \qquad + p_2 \qquad = 0 \\
	d_1 \qquad + d_3 + d_4 \qquad \qquad + p_3 = 0 \\
  \end{gather*}

  \[
	\begin{bmatrix}
	  1 & 1 & 0 & 1 & 1 & 0 & 0 \\
	  0 & 1 & 1 & 1 & 0 & 1 & 0 \\
	  1 & 0 & 1 & 1 & 0 & 0 & 1 \\
	\end{bmatrix} \begin{pmatrix}
      d_1 \\ d_2 \\ d_3 \\ d_4 \\ p_1 \\ p_2 \\ p_3
	\end{pmatrix}
	= 0
  \]
\end{frame}
%%%%%%%%%%%%%%%%%%%%
\begin{frame}{Parity-check matrix}
  \begin{definition}[Parity-check matrix]
	A matrix $H_{(n - k) \times n}$ is a parity-check matrix for an $(n,k)$ linear code $C$ if

	\[
	  C = \text{Nul}(H)
	\]
  \end{definition}

  \[
	H_{(n - k) \times n} \cdot c_{n \times 1} = 0_{(n - k) \times 1}
  \]

  \[
	\text{rank}(H) = n - k
  \]
\end{frame}
%%%%%%%%%%%%%%%%%%%%
\begin{frame}{Standard parity-check matrix}
  \begin{exampleblock}{Problem TJ--8.11}
	Parity-check matrices are NOT unique.
  \end{exampleblock}

  \centerline{Elementary row operations.}

  \vspace{0.60cm}
  \begin{definition}[Standard parity-check matrix]
	A parity-check matrix $H_{(n-k) \times n}$ is standard if

	\[
	  H_{(n-k) \times n} = \begin{bmatrix}
		A_{(n-k) \times k} \mid I_{n-k}
	  \end{bmatrix}
	\]
  \end{definition}
\end{frame}
%%%%%%%%%%%%%%%%%%%%
\begin{frame}{Generator matrix and Parity-check matrix}
  \[
	H_{(n-k) \times n} \cdot G_{n \times k} \cdot d_{k \times 1} = 0_{(n-k) \times 1}
  \]

  \begin{align*}
	H_{(n-k) \times n} &\cdot G_{n \times k} \\
	&= \begin{bmatrix}
	  A_{(n-k) \times k} \mid I_{n-k} 
	\end{bmatrix}
	\cdot
	\begin{bmatrix}
	  I_k \\ A_{(n-k) \times k}
	\end{bmatrix} \\
	&= A_{(n-k) \times k} \cdot I_k + I_{n-k} \cdot A_{(n-k) \times k} \\
	&= A_{(n-k) \times k} + A_{(n-k) \times k} \\
	&= 0_{(n-k) \times k}
  \end{align*}
\end{frame}
%%%%%%%%%%%%%%%%%%%%
\begin{frame}{Syndrome decoding}
  \[
	r = c + e_i
  \]

  \[
	r = c + (e_{i} + e_{j} + \cdots)
  \]

  \begin{definition}[Syndrome]
	\begin{align*}
	  S(r) &= H r \\
	  	&= H(c + (e_{i} + e_{j} + \cdots)) \\
		&= H (e_{i} + e_{j} + \cdots) \\
		&= H e_{i} + H e_{j} + \cdots \\
		&= S(e_i) + S{e_{j}} + \cdots
	\end{align*}
  \end{definition}
\end{frame}
%%%%%%%%%%%%%%%%%%%%
\begin{frame}{Syndrome decoding}
  \begin{exampleblock}{Problem TJ--8.13}
	\[
	  H = \begin{bmatrix}
		0 & 1 & 1 & 1 & 1 \\
		0 & 0 & 0 & 1 & 1 \\
		1 & 0 & 1 & 0 & 1
	  \end{bmatrix}
	\]

	\centerline{(d) Errors in the third and fourth bits}
  \end{exampleblock}
\end{frame}
%%%%%%%%%%%%%%%%%%%%
\begin{frame}{Extracting $d(C)$ from $H$}
  \begin{theorem}[]
	If $H$ is the parity-check matrix for a linear code $C$, then
	$d(C)$ equals the minimum number of columns of $H$ that are linearly dependent.
  \end{theorem}
\end{frame}
%%%%%%%%%%%%%%%%%%%%
\begin{frame}{Error-detecting and error-correcting capabilities}
  \begin{align*}
	t = 1 & \implies d(C) = 3 \\
	  &\iff \forall \set{c_i}, \forall \set{c_i, c_j} \text{ linearly independent} \\
	  &\iff \text{no zero column, no identical columns}
  \end{align*}
\end{frame}
%%%%%%%%%%%%%%%%%%%%
\begin{frame}{Error-detecting and error-correcting capabilities}
  \begin{exampleblock}{Problem TJ--8.21; TJ--8.23}
	If we are to use an error-correcting linear code to transmit the 128 ASCII characters,
	what size matrix must be used? What if we require only error detection?
  \end{exampleblock}

  \[
	t = 1
  \]

  \[
	2^r - 1 \ge 7 + r \implies r \ge 4
  \]
\end{frame}
%%%%%%%%%%%%%%%%%%%%
\begin{frame}{Generalized Hamming codes}
  \[
	C_{Ham} = (7,4)
  \]

  \[
	\begin{bmatrix}
	  1 & 1 & 0 & 1 & 1 & 0 & 0 \\
	  0 & 1 & 1 & 1 & 0 & 1 & 0 \\
	  1 & 0 & 1 & 1 & 0 & 0 & 1 \\
	\end{bmatrix} \begin{pmatrix}
      d_1 \\ d_2 \\ d_3 \\ d_4 \\ p_1 \\ p_2 \\ p_3
	\end{pmatrix}
	= 0
  \]
  perfect code?
  rate?
\end{frame}
%%%%%%%%%%%%%%%%%%%%
%%%%%%%%%%%%%%%%%%%%
